\section{Introduction}

In this paper we explore the bidding behavior of businesses in the California Department of Transportation (commonly known as Caltrans) auctions.
Caltrans outsources their labor for highway construction auctions by using low-bid procurement auctions.
In a low-bid procurement auction, the bidder with the lowest bid wins the auction and the price equals the lowest bid.
Within Caltrans' auctions there are two types of bidders: small business bidders and large business bidders.
Large businesses have more resources and capital, and thus have a greater advantage in winning auctions.
Therefore, to account for this, Caltrans uses bidder preference, such that small businesses will win a particular auction if they are within 5\% of a large business bidder.

We begin with a close exploration of the data, in which we further explain how Caltrans operates and analyze the
bidding data. This is done through analysis of summary statistics, plots of bidding behavior and a regression analysis of additional variables.
We have data from 705 procurement auctions held by Caltrans. This paper is broken up into
five sections, which will analyze the auction data; find the theoretically optimal
bidding functions according to the rules of Caltrans auctions; recover
an equation for estimated business costs according to that bidding function;
find estimated densities of the cost distributions according to our empirical data;
and use these estimated densities to explore if Caltrans could lower costs by
using second-price auctions (SPAs) with reserve prices.

This counterfactual was chosen because of two results. The first, the revenue
equivalence theorem, states that the expected costs/revenue are the same
for both first and second price auctions as long as they have the same method
of assigning outcomes and the same expected payment for participants.
While the 5\% margins for small businesses cannot be implemented in exactly
the same way in second-price auction, we will elaborate a system for
per-group reserve prices which should preserve the same outcomes.

Second, although reserve prices are never guaranteed to cut costs in a
procurement auction, it is possible as long as the number of auction
participants does not decrease.
Given this, it is worth studying whether Caltrans and other publicly funded
utilities which rely on private contractors could reduce their
expenditures by changing their auction choices.

