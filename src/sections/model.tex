\section{Model}

\subsection{Assumptions}

First, recall the auction rules: a small business bid
within 5\% of the lowest bid will win, and the procurer will receive their
bid. Thus, the 5\% rule affects the expectation of winning, but not the payoff.
Because there are two groups of businesses, we have asymmetric bidders which
we will refer to as \(L\) and \(S\), with \(n_L + n_S = n\), and \(n \ge 2\).

Similarly, refer to the CDFs of the costs as \(F_{L} \ne F_{S}\)
on the support  \([\ubar{c}, \bar{c}]\).

We assume that during the CalTrans auction, the bidders had costs that
were independent and drawn from an identical distribution, and that the costs
were private. (Since these were procurement auctions, cost is the analogue to
valuation). However, the distribution is conditional on some estimate.

These distributions are unknown, and we do not have complete information, on
their parameters. However, we do know that they are conditional on an engineer's
estimate provided by CalTrans, which approximates the true cost.
distributed according to some unknown, distinct parameters,
Therefore, we write that \(c \sim F_{L \text{ or } S}(\cdot \mid E)\),
where \(\cdot\) represents parameters and \(E\) represents the estimate.
 
Bidding functions are assumed to be monotonically increasing in cost.
Similarly, all bidding functions are assumed to satisfy \(\beta(0) = 0\).
Require that \( 1.05 \beta_L(\bar{c}) = \beta_S(\bar{c})\), and that
each \(\beta_{i}\) has an inverse \(\phi_{i} = \beta_{i}^{-1}\).

Finally, we assume that within the groups, each bidder uses the same optimal
strategy, denoted \(\beta_S\) and \(\beta_L\).

\subsection{Bidders' Optimization Problems}

Since the auction is a first price auction, we can define the expected payoffs
for a small business \(i\).
\( \Pi \) refers to the profit or expected payoff for \(i\).
The numbers of small/large bidders participating in the auction are
\(n_S, n_L\)). Index the  large business competitors with \(k\) and
other small business competitors with \(j \ne i\).
Recall that for each opponent, \( b_{j} = \beta_{S}(c_{j}) \) and
\( b_{k} = \beta_{L}(c_{k}) \). 
Since valuations are private, we simply refer to opponent bids \( b_j, b_k \)
in the below equation.
The probability on the right side is that all \(n_S - 1\) SB costs are higher
than their cost and that all \( n_L \) LB costs are higher than \( \frac{1}{1.05} \) their cost.
Since costs are private, we estimate this with the inverse bid functions.
\[
  \begin{aligned}
    P(b_i \text{ wins}) &= P \left(
    b_i< b_j, \, \forall j \ne i 
   \, \land \, \frac{1}{1.05} b_i < b_k \forall k \right) \\
  % \Pi(\beta_S(c_i), c_i) = \Pi(b_i, c_i) &= (b_i - c_i) \cdot
    % P \left(b_i < \beta_S(c_j) \forall j \neq i \land
    % b_i < 1.05 \beta_L(c_k) \forall k \right) \\
  \Pi(b_i, c_i) &= (b_i - c_i) \cdot
    P \left(b_i < b_j, \, \forall j \neq i \, \land \,
    b_i < 1.05 b_k, \, \forall k \right) \\
    \text{Probability all } n_S - 1, n_L & \text{ costs are higher, as the
    bid function is monotonic} \\
  \Pi(b_i, c_i) &= (b_i - c_i) \cdot \left(
    \left( 1 - F_S(\beta_{S}^{-1}(b_i) \mid E) \right)^{n_S - 1}
    \left( 1 - F_L(\beta_{L}^{-1}(\frac{1}{1.05}b_i) \mid E) \right)^{n_L} \right)
  \end{aligned}
\]

Similarly, for a large business \(i\) with bid \(b_i = \beta_L(c_i)\), with
small businesses \(j\) and other large busineses \(k \ne i\):

\[
  \begin{aligned}
    P(b_i \text{ wins}) &=
    P \left( b_i < \frac{1}{1.05} \cdot b_j, \, \forall j \, \land \,
    b_i< b_k, \, \forall k \ne i \right) \\
    % \left( 1 - F_S(1.05 \beta_{S}^{-1}(b_i)) \right)^{n_S}
    % \left( 1 - F_L(\beta_{L}^{-1}(\beta_{L}(c_i))) \right)^{n_L - 1} \right) \\
  \Pi(\beta_L (c_i), c_i) &= (b_i - c_i) \cdot
    P \left( b_i < \frac{1}{1.05} \cdot b_j, \, \forall j \, \land \,
    b_i< b_k, \, \forall k \ne i \right) \\
  &= (b_i - c_i) \cdot \left(
    \left( 1 - F_S(\beta_{S}^{-1}(1.05 b_i) \mid E) \right)^{n_S}
    \left( 1 - F_L(\beta_{L}^{-1}(b_i) \mid E) \right)^{n_L - 1} \right)
  \end{aligned}
\]
Next, we want to maximize this profit function with respect to the bid \(b_i\),
since we can manipulate the resulting first-order condition to obtain an equation
for costs in terms of bids.
Companies will choose their bid based on
\( \max_{b_i} \Pi(b_i, c_i) \).
Note that for \( y = m(x) \), we have
\[ \frac{d m^{-1}(y)}{dy} = \frac{1}{m'(m^{-1}(y))} \]

For a small business, this maximization gives the following after applying
the chain and product rule to the derivative:
\[
  \begin{aligned}
  \frac{\partial}{\partial b_i} & (b_i - c_i) \cdot \left(
    \left( 1 - F_S(\beta_{S}^{-1}(b_i) \mid E) \right)^{n_S - 1}
    \left( 1 - F_L(\beta_{L}^{-1}(\frac{b_i}{1.05}) \mid E) \right)^{n_L} \right) \\
  \implies (b_{i} - c_{i}) \cdot (
  & \left( 1 - F_L(\beta_{L}^{-1}(\frac{1}{1.05}b_i) \mid E) \right)^{n_L} \cdot
  (n_S - 1)(1 - F_S(\beta_{S}^{-1}(b_i)))^{n_S - 2} \cdot
  \frac{-f_{S}(\beta_{S}^{-1}(b_i))}{\beta_{S}'(\beta_{S}^{-1}(b_i))} \\
  + & \left( 1 - F_S(\beta_{S}^{-1}(b_i) \mid E) \right)^{n_S - 1} \cdot
  n_L (1 - F_L(\beta_{L}^{-1}(\frac{1}{1.05}b_i) \mid E) )^{n_L - 1}) \cdot \text{ (cont. next line)}\\
  (-f_{L}(\beta_{L}^{-1} & (\frac{1}{1.05}b_i) \mid E))
  \cdot \frac{1}{1.05 \beta_{L}'(\beta_{L}^{-1}(\frac{b_i}{1.05}))}) \\
  + & \left( 1 - F_S(\beta_{S}^{-1}(b_i) \mid E) \right)^{n_S - 1}
    \left( 1 - F_L(\beta_{L}^{-1}(\frac{1}{1.05}b_i) \mid E) \right)^{n_L} ) = 0
  \end{aligned}
\]
We can then move the last term over and divide through by it, giving

\[
\begin{aligned}
-1 &= (b_i - c_i) ( (n_S - 1) 
  (1 - F_S(\beta_{S}^{-1}(b_i)))^{-1} (-f_{S}(\beta_{S}^{-1}(b_i)))
  \cdot \frac{1}{\beta_{S}'(\beta_{S}^{-1}(b_i))} + \\
+ & n_L (1 - F_L(\beta_{L}^{-1}(\frac{b_i}{1.05})))^{-1} 
  (-f_{L}(\beta_{L}^{-1}(\frac{1}{1.05}b_i) \mid E))
  \cdot \frac{1}{1.05 \beta_{L}'(\beta_{L}^{-1}(\frac{b_i}{1.05}))})
\end{aligned}
\]
Taking away a \(-1\) on each side and substituting \(\beta_{S}^{-1}(b_i)\)
with \(c_i\) gives the differential equation
\[
  1 = (b_i - c_i) \cdot \left(
    \frac{(n_S - 1) \cdot f_S(c_i \mid E)}{1 - F_S(c_i \mid E) \cdot \beta_S'(c_i)} +
    \frac{(n_L) \cdot f_L(\frac{c_i}{1.05} \mid E)}{1 - F_L(\frac{c_i}{1.05} \mid E) \cdot \beta_L'(c_i)} \right)
\]
Following the same process for a large business, the first order condition is equivalent to:
\[
  1 = (b_i - c_i) \cdot \left(
    \frac{(n_S) \cdot f_S(1.05 c_i \mid E)}{1 - F_S(1.05 c_i \mid E) \cdot \beta_S'(c_i)} +
    \frac{(n_L - 1) \cdot f_L(c_i \mid E)}{1 - F_L(c_i \mid E) \cdot \beta_L'(c_i)} \right)
\]
Together we have a system of equations satisfying
\( \beta_S (\bar{c}) = 1.05 \beta_L (\bar{c}) = \bar{c}\) and
\( \beta_S( \ubar{c} ) = 1.05 \beta_L( \ubar{c} ) \).
From this system we will be able to obtain the costs and
\(F_{L \text{ or } S}(\cdot \mid E)\).
