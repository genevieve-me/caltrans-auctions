\section{Model}

\subsection{Assumptions}

We assume that during the CalTrans auction, the bidders had costs that
were independent and drawn from an identical distribution, and that the costs
were private. (Since these were procurement auctions, cost is the analogue to
valuation).
Would interdependent costs make more sense given the engineer's estimate?

Similarly, we take into account the rules of the auction: a small business bid
within 5\% of the lowest bid will win, and the procurer will receive their
bid. Thus, the 5\% rule affects the expectation of winning, but not the payoff.
Because there are two groups of businesses, we have asymmetric bidders which
we will refer to as \(L\) and \(S\).

Similarly, refer to the CDFs of the costs as \(F_{L}\) and \(F_{S}\),
with \(c \sim F_L(\cdot \mid x)\) for small bidders and
\(c \sim F_L(\cdot \mid x)\) for large bidders.
 
Bidding functions are assumed to be monotonically increasing in cost.
Similarly, all bidding functions are assumed to satisfy \(\beta(0) = 0\).
Lastly, require that each \(\beta_{i}\) satisfies \(\phi_{i} = \beta_{i}^{-1}\).

We assume that within the groups, each bidder uses the same optimal strategy,
denoted \(\beta_S\) and \(\beta_L\).

% Define a maximum cost \(\omega\) at which bidding functions should be equal
% (common highest bid).

Since the auction is a first price auction, we can define the expected payoffs
for a small business \(i\) (with large business competitors \(k\) and
other small business competitors \(j \ne i\), where the numbers of
small/large bidders are \(n_S, n_L\)) as:

\[
  \begin{aligned}
  \Pi(\beta_S (c_i), c_i) &= (\beta_S(c_i) - c_i) \cdot
    P \left( \beta_S (c_i) < \beta_S(c_j) \forall j \neq i \land
    \beta_S (c_i) < \frac{1}{1.05} \beta_L(c_k) \forall k \right) \\
  &= (\beta_S(c_i) - c_i) \cdot \left(
    \left( 1 - F_S(\beta_{S}^{-1}(\beta_S(c_i))) \right)^{n_S - 1}
    \left( 1 - F_L(\frac{1}{1.05} \beta_{L}^{-1}(\beta_L(c_i))) \right)^{n_L} \right)
  \end{aligned}
\]

Similarly, for a large business, with small businesses \(j\) and other
large busineses \(k \ne i\):

\[
  \begin{aligned}
  \Pi(\beta_L (c_i), c_i) &= (\beta_L(c_i) - c_i) \cdot
    P \left( \beta_L (c_i) < 1.05 \cdot \beta_S(c_j) \forall j \land
    \beta_L (c_i) < \beta_L(c_k) \forall k \ne i \right) \\
  &= (\beta_L(c_i) - c_i) \cdot \left(
    \left( 1 - F_S(1.05 \beta_{S}^{-1}(\beta_S(c_i))) \right)^{n_S}
    \left( 1 - F_L(\beta_{L}^{-1}(\beta_{L}(c_i))) \right)^{n_L - 1} \right)
  \end{aligned}
\]

Then, they will choose their bid based on
\[
  \max_{c_i; \beta_S} \Pi(c_i; \beta_S)
  \text{ or }
  \max_{c_i; \beta_L} \Pi(c_i; \beta_L)
\]

their first order condition.

For a small business, this equals:

\[
  1 = (\beta_L(c_i) - c_i) \cdot \left(
    \frac{(n_S - 1) \cdot f_S(c_i)}{1 - F_S(c_i) \cdot \beta_S'(c_i)} +
    \frac{(n_L - 1) \cdot \frac{1}{1.05} f_L(c_i)}{1 - F_L(c_i) \cdot \beta_L'(c_i)} \right)
\]

This equation has the boundary conditions
\(\beta_S (\bar{c}) = \beta_L (\bar{c}) = \bar{c}\) and
\( \beta_S( \ubar{c} ) = \beta_L( \ubar{c} ) = \ubar{c} \).