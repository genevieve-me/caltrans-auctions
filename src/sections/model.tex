\section{Model}

\setlength{\parindent}{0cm}

\subsection{Assumptions}

First, recall the auction rules: a small business bid
within 5\% of the lowest bid will win, and the procurer will receive their
bid. Thus, the 5\% rule affects the expectation of winning, but not the payoff.
Because there are two groups of businesses, we have asymmetric bidders which
we will refer to as \(L\) and \(S\), with \(n_L + n_S = n\), and \(n \ge 2\).

Similarly, refer to the CDFs of the costs as \(F_{L} \ne F_{S}\)
on the support  \([\ubar{c}, \bar{c}]\).

We assume that during the CalTrans auction, the bidders had costs that
were independent and drawn from an identical distribution, and that the costs
were private. (Since these were procurement auctions, cost is the analogue to
valuation). However, the distribution is conditional on some estimate.

These distributions are unknown, and we do not have complete information, on
their parameters. However, we do know that they are conditional on an engineer's
estimate provided by CalTrans, which approximates the true cost.
distributed according to some unknown, distinct parameters,
Therefore, we write that \(c \sim F_{L \text{ or } S}(\cdot \mid E)\),
where \(\cdot\) represents parameters and \(E\) represents the estimate.
 
Bidding functions are assumed to be monotonically increasing in cost.
Similarly, all bidding functions are assumed to satisfy \(\beta(0) = 0\).
Require that \(\beta_L(\bar{c}) = \beta_S(\bar{c})\), and that
each \(\beta_{i}\) has an inverse \(\phi_{i} = \beta_{i}^{-1}\).

Finally, we assume that within the groups, each bidder uses the same optimal
strategy, denoted \(\beta_S\) and \(\beta_L\).

% Define a maximum cost \(\omega\) at which bidding functions should be equal
% (common highest bid).

\subsection{Bidders' Optimization Problems}

Since the auction is a first price auction, we can define the expected payoffs
for a small business \(i\) (with large business competitors \(k\) and
other small business competitors \(j \ne i\), where the numbers of
small/large bidders are \(n_S, n_L\)) as:

% \[
%   \begin{aligned}
%   \Pi(\beta_S (c_i), c_i) &= (\beta_S(c_i) - c_i) \cdot
%     P \left( \beta_S (c_i) < \beta_S(c_j) \forall j \neq i \land
%     \beta_S (c_i) < \frac{1}{1.05} \beta_L(c_k) \forall k \right) \\
%   &= (\beta_S(c_i) - c_i) \cdot \left(
%     \left( 1 - F_S(\beta_{S}^{-1}(\beta_S(c_i))) \right)^{n_S - 1}
%     \left( 1 - F_L(\frac{1}{1.05} \beta_{L}^{-1}(\beta_L(c_i))) \right)^{n_L} \right)
%   \end{aligned}
% \]
\[
  \begin{aligned}
    P(b_i \text{ wins}) &= P \left(
    b_i< b_j \forall j \ne i 
   \land \frac{1}{1.05} b_i < b_k \forall k \right) \\
  \Pi(\beta_S(c_i), c_i) = \Pi(b_i, c_i) &= (b_i - c_i) \cdot
    P \left(b_i < \beta_S(c_j) \forall j \neq i \land
    b_i < 1.05 \beta_L(c_k) \forall k \right) \\
  \text{Since valuations } & \text{are private, denote opponent bids as } b_j, b_k \\
  \Pi(b_i, c_i) &= (b_i - c_i) \cdot
    P \left(b_i < b_j \forall j \neq i \land
    b_i < 1.05 b_k \forall k \right) \\
  &= (b_i - c_i) \cdot \left(
    \left( 1 - F_S(\beta_{S}^{-1}(b_i) \mid E) \right)^{n_S - 1}
    \left( 1 - F_L(\beta_{L}^{-1}(1.05b_i) \mid E) \right)^{n_L} \right)
  \end{aligned}
\]

Similarly, for a large business \(i\) with bid \(b_i = \beta_L(c_i)\), with
small businesses \(j\) and other large busineses \(k \ne i\):

\[
  \begin{aligned}
    P(b_i \text{ wins}) &=
    P \left( b_i < \frac{1}{1.05} \cdot b_j \forall j \land
    b_i< b_k \forall k \ne i \right) \\
    % \left( 1 - F_S(1.05 \beta_{S}^{-1}(b_i)) \right)^{n_S}
    % \left( 1 - F_L(\beta_{L}^{-1}(\beta_{L}(c_i))) \right)^{n_L - 1} \right) \\
  \Pi(\beta_L (c_i), c_i) &= (b_i - c_i) \cdot
    P \left( b_i < \frac{1}{1.05} \cdot b_j \forall j \land
    b_i< b_k \forall k \ne i \right) \\
  &= (b_i - c_i) \cdot \left(
    \left( 1 - F_S(\beta_{S}^{-1}(\frac{b_i}{1.05}) \mid E) \right)^{n_S}
    \left( 1 - F_L(\beta_{L}^{-1}(b_i) \mid E) \right)^{n_L - 1} \right)
  \end{aligned}
\]

Then, they will choose their bid based on their first order condition,
\(\max_{b_i} \Pi(b_i, c_i)\) or \(\max_{b_i} \Pi(b_i, c_i)\).

For a small business, this gives the differential equation:

% FIXME: he wants to see the full derivation

\[
  1 = (b_i - c_i) \cdot \left(
    \frac{(n_S - 1) \cdot f_S(c_i \mid E)}{1 - F_S(c_i \mid E) \cdot \beta_S'(c_i)} +
    \frac{(n_L) \cdot f_L(1.05 c_i \mid E)}{1 - F_L(1.05 c_i \mid E) \cdot \beta_L'(c_i)} \right)
\]

Similarly, for a large business the first order condition gives:

\[
  1 = (b_i - c_i) \cdot \left(
    \frac{(n_S) \cdot f_S(\frac{1}{1.05} c_i \mid E)}{1 - F_S(\frac{c_i}{1.05} \mid E) \cdot \beta_S'(c_i)} +
    \frac{(n_L - 1) \cdot f_L(c_i \mid E)}{1 - F_L(c_i \mid E) \cdot \beta_L'(c_i)} \right)
\]

FIXME: the boundary conditions need to take the 5\% rule into account.

Together we have a system of equations satisfying
\(\beta_S (\bar{c}) = \beta_L (\bar{c}) = \bar{c}\) and
\( \beta_S( \ubar{c} ) = \beta_L( \ubar{c} )\).
From this system we will be able to obtain the costs and
\(F_{L \text{ or } S}(\cdot \mid E)\).

\setlength{\parindent}{1.5em}