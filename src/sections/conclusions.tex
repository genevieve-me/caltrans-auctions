\section{Conclusion}

We were able to obtain estimates of the distribution of costs (analagous to private values)
for bidders on CalTrans procurement projects using data on their bids and estimates of
the true cost. This is because there is an optimal bidding function which we
assume the bidders followed. The work was complicated by the existence of
small- and large-business type bidders.

The first step of this analysis was to explore bidding behavior by modeling the
bidding functions for small business bidders and large business bidders.
These models demonstrated the probability of winning a particular auction,
since in first price procurement auction, the bidder with the lowest cost or value
will win the auction, assuming all participants are rational.
We then maximized the expected profits using optimization to recover the costs,
as the Caltrans data only reports submitted bids.
Finally, using conditional kernel density estimation, we found estimated
cost distributions for small business bidders and large business bidders.

Using a subset of the bidding data, as well as the data on
engineers' estimates, we plotted the PDFs and CDFs of the bidding distributions
and cost distributions for both small business bidders and large business bidders.
To simplify these calculations we only focused on auctions with the most prominent combination of small business and large business bidders.
This ends up being auctions with one small business and three large businesses.
These plots were our main results and demonstrated the respective behaviors of such bidders.

One addendum that we explored with this data was a counterfactual scenario in which
second price auction with reserve prices was used. This explored what costs may have
been for Caltrans if there was no bidder preference or `5\% rule.' Instead,
if Caltrans had set specific reserve prices for small businesses and large businesses,
we found that the average cost of Caltrans projects would have been lower.
Thus, reserve prices may be beneficial for Caltrans and other public utilities because they decrease their costs. 

This paper exhibits three main weaknesses. The first was the simplification to only look at
auctions with 1 small and 3 large business bidders, which notably reduced
the sample size from 705 to 59; and the use of the median
engineer's estimate throughout. The second was the reliance on kernel density estimation
to model bidding behavior, which is only asymptotically unbiased and presented
an issue as those density estimates were plugged in to recover cost estimates.
The last was issues with floating point error when dealing with the inverse
of very large bids, which we could see with inverse transform sampling
in the counterfactual section and with a handful of `infinite' estimates when
recovering costs, which had to be included.

Overall, the findings of this report demonstrated how firms bid in these Caltrans low-bid procurement auctions.
Using auction theory principles, we are able to evaluate and explore bidding behavior in real auctions.
These findings allow us to predict in future auctions how opposing bidders will act, so that bidders can maximize their profits.
Each case study gives us more information as to how bidders act in auctions.
% \section{Appendix - Code}

% The R code used to prepare the paper is found in `src/code/data.R`.

%\lstinputlisting[firstline=1, lastline = 32]{./code/data.R}
%\lstinputlisting{./code/data.R}
